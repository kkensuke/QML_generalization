% ---------------------------------------------------------------------------- %
% Essential Packages
% ---------------------------------------------------------------------------- %
\usepackage[T1]{fontenc}             % Ensures proper encoding for PDF output.
\usepackage{amsmath, amsfonts, amssymb, amsthm} % For advanced mathematical typesetting.
\usepackage{mathtools}               % Enhances amsmath, e.g., for \coloneqq and showonlyrefs.
\usepackage[many]{tcolorbox}         % For colored boxes around text, the 'many' option loads many libraries.
% \mathtoolsset{showonlyrefs=true}     % Number only referenced equations.


% ---------------------------------------------------------------------------- %
% Graphics and Colors
% ---------------------------------------------------------------------------- %
\usepackage{xcolor}
\usepackage{graphicx}                % For including images.
\usepackage{tikz}                    % For creating vector graphics in LaTeX.
\usepackage{quantikz}
\usetikzlibrary{3d}                  % Adds 3D functionalities to TikZ.


% ---------------------------------------------------------------------------- %
% Lists and Enumerations
% ---------------------------------------------------------------------------- %
\usepackage{enumitem}                % Control over itemize, enumerate, and description.


% ---------------------------------------------------------------------------- %
% Tables
% ---------------------------------------------------------------------------- %
\usepackage{multirow}                % Allows for multi-row cells in tables.
\usepackage{longtable}
\usepackage{booktabs}                % For professional-looking tables


% ---------------------------------------------------------------------------- %
% Other Packages
% ---------------------------------------------------------------------------- %
\usepackage{physics}                 % For simplifying notation in physics.
\usepackage{nicefrac}                % For creating "nice" fractions with \nicefrac.
\usepackage{verbatim}                % For including raw text or comments.
\usepackage{ascmac}                  % For creating boxed text.
\usepackage[linesnumbered, ruled, vlined]{algorithm2e} % For typesetting algorithms. This package should be loaded after ascmac.
\SetKwInput{kwInit}{Init}            % Custom keyword for initialization in algorithms.
\usepackage[normalem]{ulem}          % For underlining and strikeout (sout).
\usepackage{fontawesome}             % For including icons in the document.
\usepackage{wrapfig}                % For wrapping text around figures.
\usepackage[top=25truemm,bottom=20truemm,left=20truemm,right=20truemm]{geometry}                % For setting page dimensions and margins.


% ---------------------------------------------------------------------------- %
% Page Layout and Formatting
% ---------------------------------------------------------------------------- %
\allowdisplaybreaks[4]               % Allows page breaks in multiline equations.
\usepackage{float}                   % Improved control over figure and table placement.


% ---------------------------------------------------------------------------- %
% Theorems, Lemmas, etc.
% ---------------------------------------------------------------------------- %
\usepackage{apptools}                % Extends the functionality of theorem environments in appendices.
\usepackage{appendix}                % Manage appendices in the document.
\usepackage{titlesec}                % Customization of section titles.


% ---------------------------------------------------------------------------- %
% Hyperlinks
% ---------------------------------------------------------------------------- %
\usepackage[colorlinks=true,citecolor=green,linkcolor=blue]{hyperref}


% ---------------------------------------------------------------------------- %
% Theorem Environments
% ---------------------------------------------------------------------------- %
\theoremstyle{definition}
\newtheorem{theorem}{Theorem}
\newtheorem{lemma}{Lemma}
\newtheorem{corollary}{Corollary}
\newtheorem{proposition}{Proposition}
\newtheorem{example}{Example}
\newtheorem{definition}{Definition}
\newtheorem{assumption}{Assumption}
\newtheorem{remark}{Remark}
\newtheorem{problem}{Problem}
\newtheorem{property}{Property}

\AtAppendix{%
    \counterwithin{theorem}{section}
    \counterwithin{corollary}{section}
    \counterwithin{lemma}{section}
    \counterwithin{proposition}{section}
    \counterwithin{definition}{section}
    \counterwithin{assumption}{section}
    \counterwithin{remark}{section}
    \counterwithin{example}{section}
    \counterwithin{property}{section}
    \counterwithin{problem}{section}
}

% colored box for theorem environment
\tcolorboxenvironment{definition}{
    colback=gray!20!white,
    boxrule=0pt,
    boxsep=1pt,
    left=5pt,right=5pt,top=5pt,bottom=5pt,
    oversize=2pt,
    sharp corners,
    before skip=\topsep,
    after skip=\topsep,
    breakable
}
\tcolorboxenvironment{assumption}{
    colback=gray!20!white,
    boxrule=0pt,
    boxsep=1pt,
    left=5pt,right=5pt,top=5pt,bottom=5pt,
    oversize=2pt,
    sharp corners,
    before skip=\topsep,
    after skip=\topsep,
    breakable
}
\tcolorboxenvironment{proposition}{
    colback=gray!20!white,
    boxrule=0pt,
    boxsep=1pt,
    left=5pt,right=5pt,top=5pt,bottom=5pt,
    oversize=2pt,
    sharp corners,
    before skip=\topsep,
    after skip=\topsep,
    breakable
}
\tcolorboxenvironment{theorem}{
    colback=blue!20!white,
    boxrule=0pt,
    boxsep=1pt,
    left=5pt,right=5pt,top=5pt,bottom=5pt,
    oversize=2pt,
    sharp corners,
    before skip=\topsep,
    after skip=\topsep,
    breakable
}
\tcolorboxenvironment{corollary}{
    colback=blue!20!white,
    boxrule=0pt,
    boxsep=1pt,
    left=5pt,right=5pt,top=5pt,bottom=5pt,
    oversize=2pt,
    sharp corners,
    before skip=\topsep,
    after skip=\topsep,
    breakable
}
\tcolorboxenvironment{lemma}{
    colback=gray!20!white,
    boxrule=0pt,
    boxsep=1pt,
    left=5pt,right=5pt,top=5pt,bottom=5pt,
    oversize=2pt,
    sharp corners,
    before skip=\topsep,
    after skip=\topsep,
    breakable
}


% ---------------------------------------------------------------------------- %
% Comments and Revisions
% ---------------------------------------------------------------------------- %
% revision commenting tools
\newcommand{\stkout}[1]{\ifmmode\text{\sout{\ensuremath{#1}}}\else\sout{#1}\fi}
\newif\ifverbose
\verbosetrue                         % Toggle to show or hide corrections.
% ins: stuff that is new
\newcommand{\ins}[1]{\ifverbose\textcolor{blue}{#1}\else#1\fi}
% edit: stuff that has been replaced with other stuff
\newcommand{\edit}[2]{\ifverbose\textcolor{red}{\stkout{#1} #2}\else#2\fi}
% del: stuff that has been removed
\newcommand{\del}[1]{\ifverbose\textcolor{red}{\stkout{#1}}\fi}


% ---------------------------------------------------------------------------- %
% Latin abbreviations
% ---------------------------------------------------------------------------- %
\newcommand{\eg}{\text{e.g.,\ }}
\newcommand{\ie}{\text{i.e.,\ }}
\newcommand{\wrt}{\text{w.r.t.\ }}
\newcommand{\iid}{\text{i.i.d.\ }}


% ---------------------------------------------------------------------------- %
% Mathematical Shortcuts
% ---------------------------------------------------------------------------- %
% mathbb capital letters
\newcommand{\bbA}{\mathbb{A}}
\newcommand{\bbB}{\mathbb{B}}
\newcommand{\bbC}{\mathbb{C}}
\newcommand{\bbD}{\mathbb{D}}
\newcommand{\bbE}{\mathbb{E}}
\newcommand{\bbF}{\mathbb{F}}
\newcommand{\bbG}{\mathbb{G}}
\newcommand{\bbH}{\mathbb{H}}
\newcommand{\bbI}{\mathbb{I}}
\newcommand{\bbJ}{\mathbb{J}}
\newcommand{\bbK}{\mathbb{K}}
\newcommand{\bbL}{\mathbb{L}}
\newcommand{\bbM}{\mathbb{M}}
\newcommand{\bbN}{\mathbb{N}}
\newcommand{\bbO}{\mathbb{O}}
\newcommand{\bbP}{\mathbb{P}}
\newcommand{\bbQ}{\mathbb{Q}}
\newcommand{\bbR}{\mathbb{R}}
\newcommand{\bbS}{\mathbb{S}}
\newcommand{\bbT}{\mathbb{T}}
\newcommand{\bbU}{\mathbb{U}}
\newcommand{\bbV}{\mathbb{V}}
\newcommand{\bbW}{\mathbb{W}}
\newcommand{\bbX}{\mathbb{X}}
\newcommand{\bbY}{\mathbb{Y}}
\newcommand{\bbZ}{\mathbb{Z}}

% \mathrm capital letters
\newcommand{\rmA}{\mathrm{A}}
\newcommand{\rmB}{\mathrm{B}}
\newcommand{\rmC}{\mathrm{C}}
\newcommand{\rmD}{\mathrm{D}}
\newcommand{\rmE}{\mathrm{E}}
\newcommand{\rmF}{\mathrm{F}}
\newcommand{\rmG}{\mathrm{G}}
\newcommand{\rmH}{\mathrm{H}}
\newcommand{\rmI}{\mathrm{I}}
\newcommand{\rmJ}{\mathrm{J}}
\newcommand{\rmK}{\mathrm{K}}
\newcommand{\rmL}{\mathrm{L}}
\newcommand{\rmM}{\mathrm{M}}
\newcommand{\rmN}{\mathrm{N}}
\newcommand{\rmO}{\mathrm{O}}
\newcommand{\rmP}{\mathrm{P}}
\newcommand{\rmQ}{\mathrm{Q}}
\newcommand{\rmR}{\mathrm{R}}
\newcommand{\rmS}{\mathrm{S}}
\newcommand{\rmT}{\mathrm{T}}
\newcommand{\rmU}{\mathrm{U}}
\newcommand{\rmV}{\mathrm{V}}
\newcommand{\rmW}{\mathrm{W}}
\newcommand{\rmX}{\mathrm{X}}
\newcommand{\rmY}{\mathrm{Y}}
\newcommand{\rmZ}{\mathrm{Z}}

% \mathcal capital letters
\newcommand{\calA}{\mathcal{A}}
\newcommand{\calB}{\mathcal{B}}
\newcommand{\calC}{\mathcal{C}}
\newcommand{\calD}{\mathcal{D}}
\newcommand{\calE}{\mathcal{E}}
\newcommand{\calF}{\mathcal{F}}
\newcommand{\calG}{\mathcal{G}}
\newcommand{\calH}{\mathcal{H}}
\newcommand{\calI}{\mathcal{I}}
\newcommand{\calJ}{\mathcal{J}}
\newcommand{\calK}{\mathcal{K}}
\newcommand{\calL}{\mathcal{L}}
\newcommand{\calM}{\mathcal{M}}
\newcommand{\calN}{\mathcal{N}}
\newcommand{\calO}{\mathcal{O}}
\newcommand{\calP}{\mathcal{P}}
\newcommand{\calQ}{\mathcal{Q}}
\newcommand{\calR}{\mathcal{R}}
\newcommand{\calS}{\mathcal{S}}
\newcommand{\calT}{\mathcal{T}}
\newcommand{\calU}{\mathcal{U}}
\newcommand{\calV}{\mathcal{V}}
\newcommand{\calW}{\mathcal{W}}
\newcommand{\calX}{\mathcal{X}}
\newcommand{\calY}{\mathcal{Y}}
\newcommand{\calZ}{\mathcal{Z}}

% mathfrak capital letters
\newcommand{\frakA}{\mathfrak{A}}
\newcommand{\frakB}{\mathfrak{B}}
\newcommand{\frakC}{\mathfrak{C}}
\newcommand{\frakD}{\mathfrak{D}}
\newcommand{\frakE}{\mathfrak{E}}
\newcommand{\frakF}{\mathfrak{F}}
\newcommand{\frakG}{\mathfrak{G}}
\newcommand{\frakH}{\mathfrak{H}}
\newcommand{\frakI}{\mathfrak{I}}
\newcommand{\frakJ}{\mathfrak{J}}
\newcommand{\frakK}{\mathfrak{K}}
\newcommand{\frakL}{\mathfrak{L}}
\newcommand{\frakM}{\mathfrak{M}}
\newcommand{\frakN}{\mathfrak{N}}
\newcommand{\frakO}{\mathfrak{O}}
\newcommand{\frakP}{\mathfrak{P}}
\newcommand{\frakQ}{\mathfrak{Q}}
\newcommand{\frakR}{\mathfrak{R}}
\newcommand{\frakS}{\mathfrak{S}}
\newcommand{\frakT}{\mathfrak{T}}
\newcommand{\frakU}{\mathfrak{U}}
\newcommand{\frakV}{\mathfrak{V}}
\newcommand{\frakW}{\mathfrak{W}}
\newcommand{\frakX}{\mathfrak{X}}
\newcommand{\frakY}{\mathfrak{Y}}
\newcommand{\frakZ}{\mathfrak{Z}}



% mathrm
\newcommand{\const}{\mathrm{const.}}
\newcommand{\hc}{\mathrm{h.c.}}
\newcommand{\lhs}{\mathrm{(LHS)}}
\newcommand{\rhs}{\mathrm{(RHS)}}
\newcommand{\Haar}{\mathrm{Haar}}
\newcommand{\poly}{\mathrm{poly}}
\newcommand{\SWAP}{\mathrm{SWAP}}
\newcommand{\CNOT}{\mathrm{CNOT}}
\newcommand{\CZ}{\mathrm{CZ}}
\newcommand{\HS}{\mathrm{HS}}

% MathOperator
\DeclareMathOperator*{\argmin}{arg~min}
\DeclareMathOperator*{\argmax}{arg~max}
\DeclareMathOperator{\sgn}{sgn}
\DeclareMathOperator{\sign}{sign}
\DeclareMathOperator{\Supp}{Supp}
\DeclareMathOperator{\diag}{diag}
\DeclareMathOperator{\E}{\mathbb{E}}
\DeclareMathOperator{\Var}{Var}
\DeclareMathOperator{\Cov}{Cov}
\DeclareMathOperator{\Hom}{Hom}
\DeclareMathOperator{\Aut}{Aut}
\DeclareMathOperator{\End}{End}
\DeclareMathOperator*\bigcircop{\bigcirc}

% others
\renewcommand{\bar}[1]{\overline{#1}}
\renewcommand{\th}{\theta}
\newcommand{\ind}{\,\mathrm{d}}
\newcommand{\us}[2]{\underset{#2}{#1}}
\newcommand{\os}[2]{\overset{#2}{#1}}
\newcommand{\bx}{\boldsymbol{x}}
\newcommand{\by}{\boldsymbol{y}}
\newcommand{\bth}{\boldsymbol{\theta}}
\newcommand{\combi}[2]{{}_{#1}\text{C}_{#2}}
\newcommand{\defeq}{\coloneqq}
\newcommand{\dg}{^\dagger}
\newcommand{\mdg}{^{-\dagger}}
\newcommand{\T}{^\mathsf{T}}
\newcommand{\mT}{^{-\mathsf{T}}}
\newcommand{\prm}{^\prime}
\newcommand{\pd}{\partial}
\newcommand{\vt}{\,\|\,}
\newcommand{\bs}{\boldsymbol}
\newcommand{\kett}[1]{\ket*{#1}\!\rangle}
\newcommand{\braa}[1]{\langle\!\!\bra*{#1}}
\newcommand{\evv}[2]{\langle\!\langle{#1}|{#2}\rangle\!\rangle}
\newcommand{\dyadd}[1]{\ket*{#1}\!\rangle\langle\!\!\bra*{#1}}
\newcommand{\ot}{\otimes}
\newcommand{\otn}[1]{^{\otimes {#1}}}
\newcommand{\kten}[2]{\ket{#1}\otimes\ket{#2}}
\newcommand{\bten}[2]{\bra{#1}\otimes\bra{#2}}
\newcommand{\vep}{\varepsilon}
\usepackage{dsfont}
\newcommand{\bbid}{\mathds{1}}
\newcommand{\memo}[1]{\textcolor{red}{(#1)}}
% \NewDocumentCommand{\set}{m g}{%
%     \IfNoValueTF{#2}{%
%         \left\{\, #1 \,\right\}%
%     }{%
%     \left\{\, #1 ~\middle|~ #2 \,\right\}%
%     }%
% }
\NewDocumentCommand{\set}{s m o}{%
  \IfBooleanTF{#1}%
    {%
      \IfValueTF{#3}%
        {\{~ #2 ~|~ #3 ~\}}%
        {\{~ #2 ~\}}%
    }%
    {%
      \IfValueTF{#3}%
        {\left\{~ #2 ~\middle|~ #3 ~\right\}}%
        {\left\{~ #2 ~\right\}}%
    }%
}
% ---------------------------------------------------------------------------- %


% physics
% ---------------------------------------------------------------------------- %
% matrix
\newcommand{\paulii}{\mqty[\pmat{0}]}
\newcommand{\paulix}{\mqty[\pmat{1}]}
\newcommand{\pauliy}{\mqty[\pmat{2}]}
\newcommand{\pauliz}{\mqty[\pmat{3}]}

\newcommand{\rx}[1]{ \mqty[\cos \frac{#1}{2} & -i\sin\frac{#1}{2} \\ -i\sin \frac{#1}{2}& \cos\frac{#1}{2} ]}
\newcommand{\ry}[1]{ \mqty[\cos \frac{#1}{2} & -\sin\frac{#1}{2}  \\ \sin \frac{#1}{2}  & \cos\frac{#1}{2} ]}
\newcommand{\rz}[1]{ \mqty[e^{-i{#1}/2}      & 0                  \\ 0                  & e^{i{#1}/2}     ]}
\newcommand{\rot}[1]{\mqty[\cos {#1}         & -\sin {#1}         \\ \sin {#1}          & \cos {#1} ]}

\newcommand{\hadamard}{\frac{1}{\sqrt{2}}\mqty[1 & 1 \\ 1 & -1]}
\newcommand{\phaseg}{\mqty[1 & 0 \\ 0 & i]}
\newcommand{\tg}{\mqty[1 & 0 \\ 0 & e^{i\pi/4}]}
\newcommand{\mswap}{\mqty[1 & 0 & 0 & 0 \\ 0 & 0 & 1 & 0 \\ 0 & 1 & 0 & 0 \\ 0 & 0 & 0 & 1]}
\newcommand{\mcnot}{\mqty[1 & 0 & 0 & 0 \\ 0 & 1 & 0 & 0 \\ 0 & 0 & 0 & 1 \\ 0 & 0 & 1 & 0]}
\newcommand{\mcz}{\mqty[1 & 0 & 0 & 0 \\ 0 & 1 & 0 & 0 \\ 0 & 0 & 1 & 0 \\ 0 & 0 & 0 & -1]}

% sin cos
\newcommand{\sif}[2]{\sin\qty(\frac{#1}{#2})}
\newcommand{\cof}[2]{\cos\qty(\frac{#1}{#2})}

% spin
\newcommand{\up}{\uparrow} %spin up
\newcommand{\down}{\downarrow} %spin down
\newcommand{\szero}{\mqty[1 \\ 0]} %spin zero
\newcommand{\sone}{\mqty[0 \\ 1]} %spin one
% ---------------------------------------------------------------------------- %