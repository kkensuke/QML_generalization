\section{Norm}
In this section, we introduce the concept of norm and its properties. The $p$-norm extends the concept of the Euclidean norm, while the Schatten $p$-norm generalizes it to matrices.



\begin{definition}[Norm]
    Let $\calM$ be a linear space (vector space) over $\bbR$ or $\bbC$. A norm on $\calM$ is a function $\norm{\cdot} : \calM \to \bbR$ satisfying the following properties for any $\bs{x}, \bs{y} \in \calM$ and $c \in \bbR$ or $\bbC$:
    \begin{enumerate}
        \item (Positivity) $\norm{\bs{x}} \geq 0$ and $\norm{\bs{x}} = 0 \iff \bs{x} = \bs{0}$.
        \item (Homogeneity) $\norm{c\bs{x}} = |c|\norm{\bs{x}}$.
        \item (Triangle inequality) $\norm{\bs{x} + \bs{y}} \leq \norm{\bs{x}} + \norm{\bs{y}}$.
    \end{enumerate}
\end{definition}




\begin{definition}[$p$-norm or $L^p$-norm]
    Let $\bs{x} \defeq (x_1, x_2, \ldots, x_n) \in \bbR^n$ and $1 \leq p$. Then $p$-norm $\norm{\cdot}_p$ of $\bs{x}$ is defined as
    \begin{align}
        \norm{\bs{x}}_p
        &\defeq
        \begin{cases}
            \qty(\sum_{i=1}^n |x_i|^p)^\frac1p & (1 \leq p < \infty)\\[5pt]
            \max_{i} |x_i| & (p = \infty)
        \end{cases}
    \end{align}
    and normalized $p$-norm $\norm{\bs{x}}_{p,n}$ is defined as
    \begin{equation}
        \norm{\bs{x}}_{p,n} \defeq \qty(\frac1n \sum_{i=1}^n |x_i|^p)^\frac1p = \frac{1}{n^\frac1p} \norm{\bs{x}}_p
    \end{equation}
\end{definition}






\begin{lemma}[H\"{o}lder's inequality]\label{lem:holder-inequality}
    Let $1 \leq p, q \leq \infty$ and $1/p + 1/q = 1$. Then, for any $\bs{x}, \bs{y} \in \bbR^n$,
    \begin{equation}
        \qty(|\!\ev{\bs{x}|\bs{y}}\!| := \abs{\sum_{i=1}^n x_i y_i} \leq)
        \sum_{i=1}^n |x_i y_i| \leq \norm{\bs{x}}_p \norm{\bs{y}}_q
    \end{equation}
\end{lemma}



\begin{lemma}[Inequality of $p$-norms]\label{lem:inequality-p-norms}
    Let $1 \leq p \leq q$ and $\bs{x} \in \bbR^n$. The $p$-norm $\norm{\bs{x}}_p$ is decreasing in $p$, that is,
    \begin{align}
    \norm{\bs{x}}_\infty \leq \norm{\bs{x}}_q \leq \norm{\bs{x}}_p \leq \norm{\bs{x}}_1
    \end{align}
    
    And any two $p$-norms are related as
    \begin{align}
    \norm{\bs{x}}_q \leq \norm{\bs{x}}_p \leq n^{(1/p - 1/q)}\norm{\bs{x}}_q
    \end{align}
    
    In particular,
    \begin{align}
    & \norm{\bs{x}}_2 \leq\norm{\bs{x}}_1 \leq \sqrt{n}\norm{\bs{x}}_2 \\
    & \norm{\bs{x}}_{\infty} \leq\norm{\bs{x}}_2 \leq \sqrt{n}\norm{\bs{x}}_{\infty} \\
    & \norm{\bs{x}}_{\infty} \leq\norm{\bs{x}}_1 \leq n\norm{\bs{x}}_{\infty}
    \end{align}
    
    % So,
    % \begin{align}
    % \norm{\bs{x}}_{\infty} \leq \cdots \leq \norm{\bs{x}}_2 \leq\norm{\bs{x}}_1 \leq \sqrt{n}\norm{\bs{x}}_2 \leq n\norm{\bs{x}}_{\infty} .
    % \end{align}
\end{lemma}

\begin{proof}
    \quad\par
    (First inequality) For any $\bs{x} \in \bbR^n$, there exists $\bs{y} \in \bbR^n$ such that $\bs{x} = \norm{\bs{x}}_r \bs{y}$. Then, since $\norm{\bs{y}}_r = 1$
    , for any $1 \leq r < p$,
    \begin{align}
        \norm{\bs{y}}_p^p = \sum_{i=1}^n |y_i|^p \leq \sum_{i=1}^n |y_i|^r = \norm{\bs{y}}_r^r = 1 \quad (\because t^p \leq t^r \text{ for } 0 \leq t \leq 1)
        \implies \norm{\bs{y}}_p \leq 1
    \end{align}
    Therefore, $\norm{\bs{x}}_p = \norm{\bs{x}}_r\norm{\bs{y}}_p \leq \norm{\bs{x}}_r\norm{\bs{y}}_r = \norm{\bs{x}}_r$.
    
    (Second inequality) By H\"{o}lder's inequality,
    \begin{align}
        \sum\limits_{i=1}^n |a_i b_i|\leq
        \left(\sum\limits_{i=1}^n|a_i|^r\right)^{\frac{1}{r}}\left(\sum\limits_{i=1}^n|b_i|^{\frac{r}{r-1}}\right)^{1-\frac{1}{r}}
    \end{align}
    Let $|a_i|=|x_i|^p, |b_i|=1$ and $r=q/p>1$. Then
    \begin{align}
        \norm{x}_p^p = 
        \sum\limits_{i=1}^n |x_i|^p =
        \sum\limits_{i=1}^n |x_i|^p\cdot 1\leq
        \left(\sum\limits_{i=1}^n (|x_i|^p)^{\frac{q}{p}}\right)^{\frac{p}{q}}
        \left(\sum\limits_{i=1}^n 1^{\frac{q}{q-p}}\right)^{1-\frac{p}{q}}=
        \left(\sum\limits_{i=1}^n |x_i|^q\right)^{\frac{p}{q}} n^{1-\frac{p}{q}}
    \end{align}
    Therefore,
    \begin{align}
        \norm{x}_p=
        \left(\sum\limits_{i=1}^n |x_i|^p\right)^{1/p}\leq
        \left(\left(\sum\limits_{i=1}^n |x_i|^q\right)^{\frac{p}{q}} n^{1-\frac{p}{q}}\right)^{1/p}=
        \left(\sum\limits_{i=1}^n |x_i|^q\right)^{\frac{1}{q}} n^{\frac{1}{p}-\frac{1}{q}}=
        n^{1/p-1/q}\norm{x}_q
    \end{align}
\end{proof}

\begin{remark}
    Let $1 \leq r$. If $\norm{\bs{y}}_r = 1$($\iff \sum_{i=1}^n |y_i|^r = 1$), then $|y_i| \leq 1$ for any $i$. Because if there exists $y_i$ such that $1 < |y_i|$, then $\sum_{i=1}^n |y_i|^r > 1$.
\end{remark}

\begin{remark}
    From the last inequality in the proof, we have
    $\norm{x}_p \leq n^{1/p-1/q}\norm{x}_q \iff \norm{x}_{p,n} \leq \norm{x}_{q,n}$. Therefore, the normalized $p$-norm is increasing in $p$.
\end{remark}



\begin{definition}[Schatten $p$-norm]
    Let $A \in \bbC^{m \times n}$ and $1 \leq p$. Then Schatten $p$-norm of $A$ is defined as
    \begin{equation}
        \norm{A}_p = 
        \begin{cases}
            \qty(\Tr[\abs{A}^p])^{\frac1p} = \qty(\sum_{i=1}^{\min\{m,n\}} \sigma_i^p)^\frac1p & (1 \leq p < \infty) \\
            \sigma_1 & (p = \infty)
        \end{cases}
    \end{equation}
    , where $\abs{A} = \sqrt{A\dg A}$ and $\sigma_i$ is the $i$-th largest singular value of $A$ (i.e., the $i$-th largest eigenvalue of $|A|$). 
\end{definition}


\begin{remark}
\quad\par
\begin{itemize}
    \item When $p=1$, the Schatten $1$-norm is called the trace norm or nuclear norm.
    \item When $p=2$, the Schatten $2$-norm is called the Frobenius norm or Hilbert-Schmidt norm.
    \item When $p=\infty$, the Schatten $\infty$-norm is called the operator norm or spectral norm.
\end{itemize}
\end{remark}



\begin{lemma}[Properties of Schatten $p$-norm]
Let $A, B \in \calL(\calH)$ and $1 \leq p \leq q \leq \infty$. The Schatten $p$-norm has the following properties.
\begin{itemize}
    \item For any $p \leq q$, $\norm{A}_q \leq \norm{A}_p$.
    \item For any $p \in [1, \infty]$ and $U,V\in\calU(d)$, $\norm{UAV\dg}_p = \norm{A}_p$.
    \item $\norm{A}_p = \norm{A\T}_p = \norm{A^*}_p = \norm{A\dg}_p$.
    \item For any $p, q, r$ and $1/p + 1/q \leq 1/r$, $\norm{AB}_r \leq \norm{A}_p \norm{B}_q$ (H\"older's inequality).
    \begin{itemize}
        \item $\norm{AB}_1 \leq \norm{A}_1 \norm{B}_\infty$.
        \item $\norm{AB}_1 \leq \norm{A}_2 \norm{B}_2$.
    \end{itemize}
\end{itemize}
\end{lemma}






\begin{definition}[norm ball]
    Let $(\calM, \norm{\cdot})$ be a norm space.
    A norm ball centered at $\bs{x} \in \calM$ with radius $0 < \vep$ is defined as
    \begin{equation}
        B(\bs{x}, \vep, \norm{\cdot}) \defeq \set{\bs{y} \in \calM}[\norm{\bs{y}-\bs{x}} \leq \vep]
    \end{equation}
\end{definition}


\begin{remark}
    When $p$-norm is used, we call it $p$-norm ball and denote it as $B_p(\bs{x}, \vep)$.
    From the Fig. \ref{fig:p-norm-contours_1}, we can see that $B_{1}(\bs{x}, \vep) \subset B_{p}(\bs{x}, \vep) \subset B_{q}(\bs{x}, \vep) \subset B_{\infty}(\bs{x}, \vep)$ for $1 \leq p \leq q \leq \infty$.
\end{remark}

\begin{figure}[H]
    \centering
    \includegraphics[width=8cm]{p_norm_contours_1.pdf}
    \caption{The contours of $p$-norm balls with $\vep = 1$ in $\bbR^{2}$ for $p = 1, 2, 3, \infty$.}
    \label{fig:p-norm-contours_1}
\end{figure}





\begin{definition}[Diamond norm]
    Let $\calE : \calL(\calH_1) \to \calL(\calH_2)$ be a linear map. The diamond norm of $\calE$ is defined as
    \begin{align}
        \norm{\calE}_\diamond
        &\defeq \sup_{n\in\bbN}\sup_{\norm{A}_1 \leq 1} \norm{(\calE \otimes \bbid_{\bbC^{n}})(A)}_1\\
        &= \sup \set{\norm{(\calE  \otimes \bbid_{\calH_1})(\dyad{\psi}{\phi})}_1}[ \ket{\psi}, \ket{\phi} \in \calH_1 \otimes \calH_1]
    \end{align}
\end{definition}


\begin{remark}
    \quad\par
    \begin{itemize}
        \item When $\calE$ is a quantum channel, or a completely positive and trace-preserving (CPTP) map, the diamond norm is $\norm{\calE}_\diamond = 1$.
        \item If the map $\calE$ is Hermiticity-preserving (e.g. $\calE$ is the difference of two quantum channels), one can optimize over $\ket{\psi} = \ket{\phi}$ in the formula above.
        \item Refer to Ref.~\cite{kliesch2021theory} or Ref.~\cite{nechita2018almost} for further details on the diamond norm.
    \end{itemize}
\end{remark}





\begin{lemma}[Subadditivity of diamond distance \cite{wang2023enhanced}]\label{lem:subadditivity-diamond-distance}
    For any quantum channels $\calE_1, \calE_2, \calE_3, \calE_4$, where $\calE_2$ and $\calE_4$ map from $n$-qubit to $m$-qubit systems and $\calE_1$ and $\calE_3$ map from $m$-qubit to $k$-qubit systems, we have
    \begin{equation}
        \norm{\calE_1\circ\calE_2 - \calE_3\circ\calE_4}_\diamond \leq \norm{\calE_1 - \calE_3}_\diamond + \norm{\calE_2 - \calE_4}_\diamond
    \end{equation}
    , where $\norm{\cdot}_\diamond$ denotes the diamond norm of a quantum channel.
\end{lemma}


\begin{proof}
    \begin{align}
        \norm{\calE_1\circ\calE_2 - \calE_3\circ\calE_4}_\diamond
        &= \norm{\calE_1\circ\calE_2 - \calE_3\circ\calE_2 + \calE_3\circ\calE_2 - \calE_3\circ\calE_4}_\diamond \\
        &= \norm{(\calE_1 - \calE_3)\circ\calE_2} + \norm{\calE_3\circ(\calE_2 - \calE_4)}_\diamond \\
        &\leq \norm{\calE_1 - \calE_3}_\diamond\norm{\calE_2}_\diamond + \norm{\calE_3}_\diamond\norm{\calE_2 - \calE_4}_\diamond \\
        &\leq \norm{\calE_1 - \calE_3}_\diamond + \norm{\calE_2 - \calE_4}_\diamond
    \end{align}
\end{proof}



\begin{lemma}[Spectral norm and diamond norm of unitary channels \cite{wang2023enhanced}]\label{lem:unitary-channel-distance}
    Let $\calE_U(\rho) = U\rho U\dg$ and $\calE_V(\rho) = V\rho V\dg$ be unitary channels.
    Then, $$\norm{\calE_U - \calE_V}_\diamond \leq 2\norm{U - V}_\infty$$ where $\norm{\cdot}_\infty$ is the Schatten $\infty$-norm.
\end{lemma}


\begin{proof}
    For any $\ket{u}, \ket{v} \in \calH$, we have $\ket{v} = a\ket{u} + b\ket{u^\perp}$, where $\ket{u^\perp}$ is orthogonal to $\ket{u}$ and $|a|^2 + |b|^2 = 1$.
    Then,
    \begin{align}
        \dyad{u} - \dyad{v}
        &= \dyad{u} - (a\ket{u} + b\ket{u^\perp})(a^*\bra{u} + b^*\bra{u^\perp}) \\
        &= \dyad{u} - |a|^2\dyad{u} - |b|^2\dyad{u^\perp} - ab^*\dyad{u}{u^\perp} - a^*b\dyad{u^\perp}{u}\\
        &= \mqty(1-|a|^2 & -ab^* \\ -a^*b & -|b|^2)\\
        &= \mqty(|b|^2 & -ab^* \\ -a^*b & -|b|^2)
    \end{align}
    Since
    \begin{align}
        &\det(\lambda I - (\dyad{u} - \dyad{v})) = 0\\
        \iff &\lambda^2 - (|a|^2 + |b|^2)|b|^2 = 0\\
        \iff &\lambda^2 - |b|^2 = 0\\
        \iff &\lambda = \pm|b| = \pm\sqrt{1-|a|^2} = \pm\sqrt{1-|\ev{u|v}|^2}
    \end{align}
    , and trace norm is the sum of singular values, we have
    \begin{align}
    \frac12 \norm{\dyad{u} - \dyad{v}}_1
    &= \sqrt{1-|\ev{u|v}|^2} \\
    &= \sqrt{(1+|\ev{u|v}|)(1-|\ev{u|v}|)} \\
    &\leq \sqrt{2(1-\Re(\ev{u|v}))} \\
    &= \norm{\ket{u} - \ket{v}}_2
    \end{align}
    The last equality is due to the fact that
    \[
        \norm{\ket{u} - \ket{v}}_2 = \sqrt{(\bra{u} - \bra{v})(\ket{u} - \ket{v})} = \sqrt{2 - \ev{u|v} - \ev{v|u}} = \sqrt{2(1-\Re(\ev{u|v}))}.
    \]
    
    The diamond distance bound then is a direct consequence of this relation.
    \begin{align}
    \frac{1}{2}\norm{\calE_U - \calE_V}_\diamond
    &= \sup_{\dyad{\psi}} \frac{1}{2} \norm{\calE_U(\dyad{\psi}) - \calE_V(\dyad{\psi})}_1 \\
    &\leq \sup_{\ket{\psi}} \norm{(U-V)\ket{\psi}}_2 \\
    &= \text{largest singular value of } U-V \\
    &= \norm{U-V}_\infty
    \end{align}
\end{proof}



The following lemma translates the distance of rotation operators to the distance of their corresponding angles.

\begin{lemma}[\cite{wang2023enhanced}]\label{lem:rotation-operator-distance}
    Given an arbitrary Pauli matrix $P \in$ $\{I, X, Y, Z\}$ and two arbitrary angles $\th$ and $\tilde{\th}$, the corresponding 1-qubit rotation operators are $R(\th)=e^{-\frac{i}{2} \th P}$ and $R(\tilde{\th})=e^{-\frac{i}{2} \tilde{\th} P}$, respectively. Then, the distance between the two operators measured by the Schatten $\infty$-norm can be upper bounded as
    \begin{equation}
        \|R(\th)-R(\tilde{\th})\|_{\infty} \leq \frac{1}{2}|\th-\tilde{\th}|
    \end{equation}
\end{lemma}


\begin{proof}
    According to the definition of rotation operators, we have
    $R(\th)-R(\tilde{\th})=\left(\cos \frac{\th}{2}-\cos \frac{\tilde{\th}}{2}\right) I-i\left(\sin \frac{\th}{2}-\sin \frac{\tilde{\th}}{2}\right) P$,
    whose singular value is $2 \sin \frac{\th-\tilde{\th}}{4}$ with the multiplicity 2. Thus,
    \begin{equation}
        \|R(\th)-R(\tilde{\th})\|_{\infty}=2 \abs{\sin \frac{\th-\tilde{\th}}{4}} \leq \frac{1}{2}|\th-\tilde{\th}|
    \end{equation}
\end{proof}